% Author: Dominik Harmim <harmim6@gmail.com>

\documentclass[a4paper, 11pt]{scrartcl}

\usepackage[czech]{babel}
\usepackage[utf8]{inputenc}
\usepackage[T1]{fontenc}
\usepackage{times}
\usepackage[left=2cm, top=3cm, text={17cm, 24cm}]{geometry}
\usepackage[unicode, colorlinks, hypertexnames=false, citecolor=red]{hyperref}
\usepackage{fancyhdr}
\usepackage{lastpage}
\usepackage{tikz}
\usepackage{makecell}
\usepackage[shortlabels]{enumitem}
\usepackage{amssymb}

\newcommand{\NUMBER}{1}
\newcommand{\COURSE}{Teoretická informatika (TIN)}
\newcommand{\AUTHOR}{Dominik Harmim\,--\,xharmi00}

\newcommand*{\QEDB}{\hfill\ensuremath{\square}}

\pagestyle{fancy}
\fancyhead[L]{\AUTHOR}
\fancyhead[C]{\COURSE}
\fancyhead[R]{\today}

\fancyfoot[C]{}
\fancyfoot[R]{\thepage\,/\,\pageref*{LastPage}}


\begin{document}
    \begin{center}
        {\Large Úkol~\NUMBER}
    \end{center}


    \section*{1.~příklad}

    Uvažujme operaci~$ \circ $~definovanou následovně: $ L_1 \circ L_2 = L_1
    \cup \overline{L_2} $. S využitím uzávěrových vlastností dokažte, nebo
    vyvraťte, následující vztahy:
    \begin{enumerate}[(a)]
        \item
            $ L_1, L_2 \in \mathcal{L}_3 \Rightarrow L_1 \circ L_2 \in
            \mathcal{L}_3 $

        \item
            $ L_1 \in \mathcal{L}_3, L_2 \in \mathcal{L}_2^D \Rightarrow L_1
            \circ L_2 \in \mathcal{L}_2^D $

        \item
            $ L_1 \in \mathcal{L}_3, L_2 \in \mathcal{L}_2 \Rightarrow L_1
            \circ L_2 \in \mathcal{L}_2 $
    \end{enumerate}
    $ \mathcal{L}_2^D $~značí třídu deterministických bezkontextových jazyků,
    $ \mathcal{L}_2 $~třídu bezkontextových jazyků a~$ \mathcal{L}_3 $~třídu
    regulárních jazyků.

    \subsection*{Řešení:}

    \begin{enumerate}[(a)]
        \item
            \textbf{Vztah~(a)~je platný.} \\
            Důkaz:
            \begin{itemize}
                \item
                    Vztah přepíšeme s~využitím definované operace~$ \circ $: \\
                    $ L_1, L_2 \in \mathcal{L}_3 \Rightarrow (L_1 \circ L_2 \in
                    \mathcal{L}_3) \Leftrightarrow (L_1 \cup \overline{L_2} \in
                    \mathcal{L}_3) $

                \item
                    Podle věty~3.23\footnote{\label{tin}Studijní text\,--\,%
                        \url{https://www.fit.vutbr.cz/study/courses/TIN/public/Texty/TIN-studijni-text.pdf}%
                    .}
                    třída regulárních jazyků~$ \mathcal{L}_3 $~tvoří množinovou
                    \emph{Booleovu algebru}, z~čehož plyne uzavřenost této
                    třídy vůči doplňku a~sjednocení.

                \item
                    Díky uzavřenosti vůči doplňku platí vtah $ L_2 \in
                    \mathcal{L}_3 \Rightarrow \overline{L_2} \in
                    \mathcal{L}_3 $.

                \item
                    Konečně díky uzavřenosti vůči sjednocení platí vztah $ L_1
                    \in \mathcal{L}_3 \Rightarrow L_1 \cup \overline{L_2} \in
                    \mathcal{L}_3 $.

                \item
                    Potom tedy platí i~vztah $ L_1, L_2 \in \mathcal{L}_3
                    \Rightarrow L_1 \cup \overline{L_2} \in \mathcal{L}_3 $,
                    který je ekvivalentní vztahu~(a). \QEDB
            \end{itemize}

        \item
            \textbf{Vztah~(b)~je platný.} \\
            Důkaz:
            \begin{itemize}
                \item
                    Vztah přepíšeme s~využitím definované operace~$ \circ $: \\
                    $ L_1 \in \mathcal{L}_3, L_2 \in \mathcal{L}_2^D \Rightarrow
                    (L_1 \circ L_2 \in \mathcal{L}_2^D) \Leftrightarrow (L_1
                    \cup \overline{L_2} \in \mathcal{L}_2^D) $

                \item
                    Podle věty~4.27\footref{tin} jsou deterministické
                    bezkontextové jazyky~$ \mathcal{L}_2^D $~uzavřeny vůči
                    doplňku. Díky této větě platí i~vztah $ L_2 \in
                    \mathcal{L}_2^D \Rightarrow \overline{L_2} \in
                    \mathcal{L}_2^D $.

                \item
                    Věta~4.27\footref{tin} také říká, že deterministické
                    bezkontextové jazyky~$ \mathcal{L}_2^D $~jsou uzavřeny
                    vůči průniku s~regulárními jazyky~$ \mathcal{L}_3 $. Podle
                    této věty tedy platí následující vztah: \\ $ L_1 \in
                    \mathcal{L}_3, L_2 \in \mathcal{L}_2^D \Rightarrow (L_1
                    \cap \overline{L_2} \in \mathcal{L}_2^D) \Leftrightarrow
                    (\overline{L_1 \cap \overline{L_2}}\in \mathcal{L}_2^D) $

                \item
                    S~využitím \emph{De Morganova zákona} lze tento vztah
                    upravit na $  L_1 \in \mathcal{L}_3, L_2 \in
                    \mathcal{L}_2^D \Rightarrow \overline{L_1} \cup
                    \overline{\overline{L_2}} \in \mathcal{L}_2^D $.

                \item
                    S~využitím věty~4.27\footref{tin} a~věty~3.23\footref{tin},
                    která říká, že třída regulárních jazyků~$ \mathcal{L}_3
                    $~tvoří množinovou \emph{Booleovu algebru}, z~čehož plyne
                    uzavřenost této třídy vůči doplňku, můžeme výše uvedený
                    vztah dále upravit na tvar $ L_1 \in \mathcal{L}_3, L_2 \in
                    \mathcal{L}_2^D \Rightarrow L_1 \cup \overline{L_2} \in
                    \mathcal{L}_2^D $, což je vztah ekvivalentní zadanému
                    vztahu~(b). \QEDB
            \end{itemize}

        \item
            \textbf{Vztah~(c)~není platný.} \\
            Důkaz sporem:
            \begin{itemize}
                \item
                    Předpokládejme, že vztah~(c)~je platný.

                \item
                    Vztah přepíšeme s~využitím definované operace~$ \circ $: \\
                    $ L_1 \in \mathcal{L}_3, L_2 \in \mathcal{L}_2 \Rightarrow
                    (L_1 \circ L_2 \in \mathcal{L}_2) \Leftrightarrow (L_1
                    \cup \overline{L_2} \in \mathcal{L}_2) $

                \item
                    Podle věty~2.4\footref{tin} platí vztah $ \mathcal{L}_3
                    \subset \mathcal{L}_2 $, proto můžeme výše uvedený vztah
                    upravit na \\ $ L_2 \in \mathcal{L}_2 \Rightarrow
                    \overline{L_2} \in \mathcal{L}_2 $.

                \item
                    Věta~4.24\footref{tin} však říká, že bezkontextové
                    jazyky~$ \mathcal{L}_2 $~nejsou uzavřeny vůči doplňku,
                    tj. \\ $ L_2 \in \mathcal{L}_2 \Rightarrow \overline{L_2}
                    \notin \mathcal{L}_2 $, což je \textbf{spor}.
                    \textbf{Vztah~(c)~tedy neplatí.} \QEDB
            \end{itemize}
    \end{enumerate}


    \section*{2.~příklad}

    Mějme jazyk~$ L $~nad abecedou $ \{a, b, \#\} $ definovaný následovně:
    $ L = \{a^ib^j\#a^kb^l\ |\ i + 2j = 2k + l\} $. Sestrojte deterministický
    zásobníkový automat~$ M_L $~takový, že $ L(M_L) = L $.

    \subsection*{Řešení:}

    $ M_L = (Q, \Sigma, \Gamma, \delta, q_0, Z_0, F) = (\{q_0, q_1,
    q_2, q_3, q_4, q_5\}, \{a, b, \#\}, \{\sim, *, \circ\}, \delta, q_0, \sim,
    \{q_5\}) $, kde \\ $ \delta \subseteq Q \times (\Sigma \cup \{\epsilon\})
    \times \Gamma \rightarrow 2^{Q \times \Gamma^*} $ je přechodová funkce
    definovaná následovně: \\ \\
    \begin{minipage}{.45\linewidth}
        \begin{itemize}[label={}]
            \item $ \delta(q_0, a, \sim) = (q_0, *\!\sim) $
            \item $ \delta(q_0, a, *) = (q_0, \circ*) $
            \item $ \delta(q_0, a, \circ) = (q_0, \circ\circ) $
            \item $ \delta(q_0, b, \sim) = (q_1, \circ*\!\sim) $
            \item $ \delta(q_0, b, *) = (q_1, \circ\!\circ\!*) $
            \item $ \delta(q_0, b, \circ) = (q_1, \circ\!\circ\!\circ) $
            \item $ \delta(q_0, \#, \sim) = (q_5, \epsilon) $
            \item $ \delta(q_0, \#, *) = (q_2, *) $
            \item $ \delta(q_0, \#, \circ) = (q_2, \circ) $
            \item $ \delta(q_1, b, \circ) = (q_1, \circ\!\circ\!\circ) $
        \end{itemize}
    \end{minipage}
    \hfill
    \begin{minipage}{.45\linewidth}
        \begin{itemize}[label={}]
            \item $ \delta(q_1, \#, \circ) = (q_2, \circ) $
            \item $ \delta(q_2, a, \circ) = (q_3, \epsilon) $
            \item $ \delta(q_2, b, \circ) = (q_4, \epsilon) $
            \item $ \delta(q_2, b, *) = (q_5, \epsilon) $
            \item $ \delta(q_3, \epsilon, \circ) = (q_2, \epsilon) $
            \item $ \delta(q_3, \epsilon, *) = (q_5, \epsilon) $
            \item $ \delta(q_4, b, \circ) = (q_4, \epsilon) $
            \item $ \delta(q_4, b, *) = (q_5, \epsilon) $
            \item $ \delta(q_5, \epsilon, \sim) = (q_5, \epsilon) $
        \end{itemize}
    \end{minipage} \\ \bigbreak
    \textbf{Grafická reprezentace automatu~$ M_L $}:
    \begin{center}
        \begin{tikzpicture}[scale=.114]
            \tikzstyle{every node}+=[inner sep=0pt]
            \draw [black] (11.1,-36.9) circle (3);
            \draw (11.1,-36.9) node {$ q_0 $};
            \draw [black] (11.1,-10.3) circle (3);
            \draw (11.1,-10.3) node {$ q_1 $};
            \draw [black] (37.5,-36.9) circle (3);
            \draw (37.5,-36.9) node {$ q_2 $};
            \draw [black] (56.4,-10.3) circle (3);
            \draw (56.4,-10.3) node {$ q 3$};
            \draw [black] (56.4,-36.9) circle (3);
            \draw (56.4,-36.9) node {$ q_4 $};
            \draw [black] (72.9,-36.9) circle (3);
            \draw (72.9,-36.9) node {$ q_5 $};
            \draw [black] (72.9,-36.9) circle (2.4);

            \draw [black] (2.9,-36.9) -- (8.1,-36.9);
            \fill [black] (8.1,-36.9) -- (7.3,-36.4) -- (7.3,-37.4);
            \draw (2.4,-36.9) node [left] {$ \sim $};

            \draw [black] (10.935,-39.884) arc (24.57557:-263.42443:2.25);
            \fill [black] (8.63,-38.58) -- (7.7,-38.46) -- (8.11,-39.37);
            \draw (5.96,-43.28) node [below] {\makecell[l]{$ a,
                \circ/\!\circ\!\circ $ \\ $ a, */\!\circ\!* $ \\ $ a,
                \sim\!/*\!\sim $}};

            \draw [black] (9.777,-7.62) arc (234:-54:2.25);
            \fill [black] (12.42,-7.62) -- (13.3,-7.27) -- (12.49,-6.68);
            \draw (11.1,-3.05) node [above] {$ b,
                \circ/\!\circ\!\circ\circ $};

            \draw [black] (14.1,-36.9) -- (34.5,-36.9);
            \fill [black] (34.5,-36.9) -- (33.7,-36.4) -- (33.7,-37.4);
            \draw (24.3,-37.4) node [below] {\makecell[l]{$ \#, */* $ \\ $ \#,
                \circ/\circ $}};

            \draw [black] (11.1,-33.9) -- (11.1,-13.3);
            \fill [black] (11.1,-13.3) -- (10.6,-14.1) -- (11.6,-14.1);
            \draw (10,-24.6) node [left] {\makecell[l]{$ b,
                \circ/\!\circ\!\circ\circ $ \\ $ b, */\!\circ\!\circ* $
                \\ $ b, \sim\!/\!\circ\!*\!\sim $}};

            \draw [black] (71.515,-39.56) arc (-30.02454:-149.97546:34.089);
            \fill [black] (71.52,-39.56) -- (70.68,-40) -- (71.55,-40.5);
            \draw (42,-58.09) node [below] {$ \#, \sim\!/\epsilon $};

            \draw [black] (13.21,-12.43) -- (35.39,-34.77);
            \fill [black] (35.39,-34.77) -- (35.18,-33.85) -- (34.47,-34.56);
            \draw (24.82,-26) node [left] {$ \#, \circ/\circ $};

            \draw [black] (55.707,-13.218) arc (-15.50988:-55.27971:39.795);
            \fill [black] (55.71,-13.22) -- (55.01,-13.86) -- (55.98,-14.12);
            \draw (54,-21) node [right] {$ a, \circ/\epsilon $};

            \draw [black] (40.5,-36.9) -- (53.4,-36.9);
            \fill [black] (53.4,-36.9) -- (52.6,-36.4) -- (52.6,-37.4);
            \draw (46.95,-37.4) node [below] {$ b, \circ/\epsilon $};

            \draw [black] (70.933,-39.162) arc (-44.87981:-135.12019:22.203);
            \fill [black] (70.93,-39.16) -- (70.01,-39.38) -- (70.72,-40.08);
            \draw (55.2,-46.2) node [below] {$ b, */\epsilon $};

            \draw [black] (38.003,-33.943) arc (167.85168:121.35873:34.493);
            \fill [black] (38,-33.94) -- (38.66,-33.27) -- (37.68,-33.06);
            \draw (43.01,-19.85) node [left] {$ \epsilon, \circ/\epsilon $};

            \draw [black] (59.093,-11.619) arc (60.73325:2.88946:27.11);
            \fill [black] (72.91,-33.9) -- (73.37,-33.08) -- (72.38,-33.13);
            \draw (69.51,-19.69) node [right] {$ \epsilon, */\epsilon $};

            \draw [black] (55.077,-34.22) arc (234:-54:2.25);
            \fill [black] (57.72,-34.22) -- (58.6,-33.87) -- (57.79,-33.28);
            \draw (56.4,-29.65) node [above] {$ b, \circ/\epsilon $};

            \draw [black] (59.4,-36.9) -- (69.9,-36.9);
            \fill [black] (69.9,-36.9) -- (69.1,-36.4) -- (69.1,-37.4);
            \draw (64.65,-37.4) node [below] {$ b, */\epsilon $};

            \draw [black] (75.127,-38.893) arc (75.90462:-212.09538:2.25);
            \fill [black] (72.67,-39.88) -- (71.99,-40.53) -- (72.96,-40.78);
            \draw (76.38,-43.68) node [below] {$ \epsilon, \sim\!/\epsilon $};
        \end{tikzpicture}
    \end{center}


    \section*{3.~příklad}

    Dokažte, že jazyk~$ L $~z~předchozího příkladu není regulární.

    \subsection*{Řešení:}

    Důkaz sporem:
    \begin{itemize}
        \item
            Předpokládejme, že jazyk~$ L $~ je regulární, tj. $ L \in
            \mathcal{L}_3$.

        \item
            Potom dle \emph{Pumping lemma} (věta~3.18\footref{tin}) platí: \\
            $ \exists\,k > 0: \forall\,w \in L: |w| \geq k \Rightarrow \\
            \Rightarrow (\exists\,x, y, z \in \{a, b, \#\}^*: w = xyz \wedge y
            \neq \epsilon \wedge |xy| \leq k \wedge \forall\,i \geq 0: xy^iz \in
            L ) $

        \item
            Uvažme libovolné~$ k $~splňující výše uvedené.

        \item
            Zvolme $ w = a^k\#b^k $. $ w \in L $, protože $ L = \{a^{i^\prime}%
            b^{j^\prime}\#a^{k^\prime}b^{l^\prime}\ |\ i^\prime + 2j^\prime =
            2k^\prime + l^\prime\} $ a~$ w = a^kb^0\#a^0b^k $, kde platí
            $ (k + 2 \cdot 0 = 2 \cdot 0 + k) \Rightarrow (k = k) $. Dále
            platí, že $ |w| = 2k + 1 \geq k $.

        \item
            Tedy $ \exists\,x, y, z \in \{a, b, \#\}^*: a^k\#b^k = xyz \wedge
            y \neq \epsilon \wedge |xy| \leq k \wedge \forall\,i \geq 0:
            xy^iz \in L $.

        \item
            Uvažme libovolné $ x, y, z $ vyhovující výše uvedené podmínce.

        \item
            Z~ toho, že $ |xy| \leq k $ a~$ y \neq \epsilon $ plyne:
            $ x = a^r \wedge y = a^s \wedge z = a^{k-r-s}\#b^k \wedge
            r \geq 0 \wedge s > 0 \wedge r + s \leq k$.

        \item
            Zvolme $ i = 2 $, potom $ xy^2z = a^ra^sa^sa^{k-r-s}\#b^k =
            a^{k+s}\#b^k \notin L $.

        \item
            To je \textbf{spor}, protože $ k + s \neq k $, protože $ s > 0 $
            a~z~\emph{Pumping lemma} plyne, že $ xy^iz \in L $. Proto
            \textbf{jazyk~$ L $~není regulární}, tj. $ L \notin \mathcal{L}_3 $.
            \QEDB
    \end{itemize}
\end{document}
