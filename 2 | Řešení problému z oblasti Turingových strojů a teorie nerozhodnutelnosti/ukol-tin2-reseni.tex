% Author: Dominik Harmim <harmim6@gmail.com>

\documentclass[a4paper, 11pt]{scrartcl}

\usepackage[czech]{babel}
\usepackage[utf8]{inputenc}
\usepackage[T1]{fontenc}
\usepackage{times}
\usepackage[left=2cm, top=3cm, text={17cm, 24cm}]{geometry}
\usepackage[unicode, colorlinks, hypertexnames=false, citecolor=red]{hyperref}
\usepackage{fancyhdr}
\usepackage{lastpage}
\usepackage[shortlabels]{enumitem}
\usepackage{amssymb}
\usepackage{amsmath}
\usepackage{newtxtext, newtxmath}
\usepackage{listings}

\newcommand{\NUMBER}{2}
\newcommand{\COURSE}{Teoretická informatika (TIN)}
\newcommand{\AUTHOR}{Dominik Harmim\,--\,xharmi00}

\newcommand*{\QEDB}{\hfill\ensuremath{\square}}

\lstset{
    basicstyle=\ttfamily,
    keywordstyle=\color{blue},
    numbers=left,
    tabsize=2,
    frame=shadowbox,
    firstnumber=0
}

\pagestyle{fancy}
\fancyhead[L]{\AUTHOR}
\fancyhead[C]{\COURSE}
\fancyhead[R]{\today}

\fancyfoot[C]{}
\fancyfoot[R]{\thepage\,/\,\pageref*{LastPage}}

\setlength{\parindent}{0pt}


\begin{document}
	\begin{center}
		{\Large Úkol~\NUMBER}
	\end{center}


	\section*{1.~příklad}

	Uvažujte jazyk $ L = \{w \in \{a, b\}^*\ |\ \#_a(w) = \#_b(w)\} $, kde
	$ \#_x(w) $ značí počet výskytů symbolů~$ x $ v~řetězci~$ w $. Dokažte,
	že jazyk~$ L $ je bezkontextový. Postupujte následovně:
	\begin{enumerate}[(a)]
		\item
			Nejdříve navrhněte gramatiku~$ G $, která bude mít za cíl
			jazyk~$ L $ generovat.

		\item
			Poté pomocí indukce k~délce slova $ w \in L $ dokažte, že $ L =
			L(G) $.
	\end{enumerate}

	\subsection*{Řešení:}

	\begin{enumerate}[(a)]
		\item
			Nechť~$ \boldsymbol{G} $ je následující gramatika generující
			jazyk~$ L $: \\
			$ \boldsymbol{G} = (\{S\}, \{a, b\}, \{S \rightarrow aSbS\ |\
			bSaS\ |\ \varepsilon\}, S) $

		\item
			Důkaz, že $ \boldsymbol{L = L(G)} $ provedeme dokázáním, že i)
			$ \boldsymbol{L(G) \subseteq L} $ a~ii) $ \boldsymbol{L \subseteq
			L(G)} $ matematickou indukcí k~délce slova $ w \in L $.

			\begin{enumerate}[i)]
				\item
					Důkaz, že $ \boldsymbol{L(G) \subseteq L} $.

					\begin{enumerate}[a)]
						\item
							Důkaz vzhledem k~délce slova $ i = 0 $.

							\begin{itemize}[label=$ \bullet $]
								\item
									Slovo o~délce~$ 0 $, $ \varepsilon:
									|\varepsilon| = 0 $ lze vygenerovat
									pravidlem $ S \rightarrow \varepsilon $
									gramatiky~$ G $, tj. $ \varepsilon \in
									L(G) $.

								\item
									Zároveň platí, že $ \varepsilon \in L $.

								\item
									Pro $ i = 0 $ dokazované tvrzení tedy platí.
							\end{itemize}

						\item
							Předpokládejme, že pro slovo~$ w $ platí, že
							$ |w| \leq i \wedge S \Rightarrow^* w : w \in L $.
							Na základě tohoto indukčního předpokladu ukažme,
							že dokazované tvrzení platí i~pro slova délky
							$ i + 2 $.

							\begin{itemize}[label=$ \bullet $]
								\item
									Gramatika~$ G $ derivuje následující
									řetězce $ S \Rightarrow aSbS
									\Rightarrow^* aw^{\prime}bw^{\prime
									\prime} = w_1 $, $ S \Rightarrow bSaS
									\Rightarrow^* bw^{\prime}aw^{\prime
									\prime} = w_2 $.

								\item
									Dle indukčního předpokladu platí, že když
									$ S \Rightarrow^* w^\prime \wedge S
									\Rightarrow^* w^{\prime\prime} $, tak
									$ w^\prime \in L \wedge w^{\prime\prime}
									\in L $, protože $ w^\prime \in L(G)
									\wedge w^{\prime\prime} \in L(G) \wedge
									|w^\prime| \leq i \wedge |w^{\prime
									\prime}| \leq i $, a~tedy $ w^\prime,
									w^{\prime\prime} \in \{a, b\}^*:
									\#_a(w^\prime) = \#_b(w^\prime)
									\wedge \#_a(w^{\prime\prime}) = \#_b(w^{
									\prime\prime}) $.

								\item
									Z~výše uvedeného plyne, že $ \#_a(w_1)
									= \#_a(w^\prime) + \#_a(w^{\prime\prime})
									+ 1 = \#_b(w^\prime) + \#_b(w^{\prime
									\prime}) + 1 = \#_b(w_1) \Rightarrow
									w_1 \in L. $

								\item
									A~zároveň platí, že $ \#_a(w_2)
									= \#_a(w^\prime) + \#_a(w^{\prime\prime})
									+ 1 = \#_b(w^\prime) + \#_b(w^{\prime
									\prime}) + 1 = \#_b(w_2) \Rightarrow
									w_2 \in L. $
							\end{itemize}

						\item
							$ \boldsymbol{L(G) \subseteq L} $ \textbf{tedy
							platí.}
					\end{enumerate}

				\item
					Důkaz, že $ \boldsymbol{L \subseteq L(G)} $.

					\begin{enumerate}[a)]
						\item
							Důkaz vzhledem k~délce slova $ i = 0 $.

							\begin{itemize}[label=$ \bullet $]
								\item
									Slovo o~délce~$ 0 $, $ \varepsilon:
									|\varepsilon| = 0 \wedge \varepsilon
									\in L $.

								\item
									Zároveň~$ \varepsilon $ lze vygenerovat
									gramatikou~$ G $ použitím pravidla $ S
									\rightarrow \varepsilon $, tj.
									$ \varepsilon \in L(G) $.

								\item
									Pro $ i = 0 $ tedy dokazované tvrzení
									platí.
							\end{itemize}

						\item
							Předpokládejme, že pro slovo~$ w $ platí, že
							$ |w| \leq i \wedge w \in L : S \Rightarrow^*
							w $. Na základě tohoto indukčního předpokladu
							ukážeme, že dokazované tvrzení platí i~pro
							slova délky $ i + 2 $.

							\begin{itemize}[label=$ \bullet $]
								\item
									Pro řetězce z~množiny $ W = \{w \in
									L\ |\ \#_a(w) = \frac{i + 2}{2} \wedge
									\#_b(w) = \frac{i + 2}{2} \wedge |w| =
									i + 2\} $ platí, že $ \forall\ w
									\in W : \exists\ w^\prime \in L :
									\#_a(w) = \#_a(w^\prime) + 1 =
									\#_b(w^\prime) + 1 = \#_b(w) \wedge
									|w^\prime| \leq i $.

								\item
									Dle indukčního předpokladu tedy platí,
									že $ \forall\ w \in W : \exists\
									w^\prime \in L : S \rightarrow
									w^\prime $.

								\item
									Řetězce z~množiny~$ W $ lze generovat
									gramatikou~$ G $ následovně:
									\begin{itemize}[label=]
										\item
											$ S \Rightarrow aSbS \Rightarrow
											aSb \Rightarrow aw^{\prime}b $

										\item
											$ S \Rightarrow aSbS \Rightarrow
											abS \Rightarrow abw^{\prime} $

										\item
											$ S \Rightarrow bSaS \Rightarrow
											bSa \Rightarrow bw^{\prime}a $

										\item
											$ S \Rightarrow bSaS \Rightarrow
											baS \Rightarrow baw^{\prime} $
									\end{itemize}

								\item
									Všechny výše uvedené řetězce patří do
									jazyka~$ L $, protože $ w^\prime \in L $,
									jak bylo ukázáno výše.
							\end{itemize}

						\item
							$ \boldsymbol{L \subseteq L(G)} $ \textbf{tedy
							platí.}
					\end{enumerate}
			\end{enumerate}
			\QEDB
	\end{enumerate}


	\section*{2.~příklad}

	Uvažujte \emph{doprava čtený jazyk} TS~$ M $, značený jako~$ L^P(M) $,
	který je definován jako množina řetězců, které~$ M $ přijme v~běhu, při
	kterém nikdy nepohne hlavou \emph{doleva} a~nikdy nepřepíše žádný symbol
	na pásce za jiný. Dokažte, zda je problém prázdnosti doprava čteného
	jazyka TS~$ M $, tj. zda~$ L^P(M) = \varnothing $, rozhodnutelný:
	\begin{itemize}
		\item
			pokud \emph{ano}, napište algoritmus v~pseudokódu, který daný
			problém bude rozhodovat;

		\item
			pokud \emph{ne}, dokažte nerozhodnutelnost redukcí z~jazyka
			$ HP $.
	\end{itemize}

	\subsection*{Řešení:}

	Problém prázdnosti doprava čteného jazyka Turingova stroje~$ M $, tj.
	$ L^P(M) = \varnothing $, \textbf{je rozhodnutelný}. Důkazem nechť
	je následující algoritmus, který tento problém rozhoduje.

	\textbf{Algoritmus:}
	\begin{addmargin}[10pt]{0pt}
		\underline{Vstup:}
			Deterministický Turingův stroj $ M = (Q, \Sigma, \Gamma,
			\delta, q_0, q_f) $ s~přechodovou parciální funkcí~$ \delta $
			definovanou následovně: $ (Q \setminus \{q_f\}) \times
			\Gamma \rightarrow Q \times (\Gamma \cup \{L, R\}) $, kde
			$ L, R \notin \Gamma $.
		\\[5pt]
		\underline{Výstup:}
			$
				\begin{cases}
					\mathtt{TRUE} &
					\text{pokud}\ \ L^P(M) = \varnothing \\
					\mathtt{FALSE} &
					\text{jinak, tj. pokud}\ \ L^P(M) \neq \varnothing
				\end{cases}
			$
		\\[5pt]
		\underline{Metoda:}
			\begin{enumerate}
				\item
					Nechť $ R_\delta \subseteq Q \times Q $ je binární
					relace, která popisuje, zda je v~Turingově
					stroji~$ M $ možný přímý přechod (definovaný
					jazykem~$ L^P(M) $) mezi danou dvojicí stavů $ (p,
					q) $, definována na základě přechodové funkce~$
					\delta $ následovně: \\
					$ R_\delta = \{(p, q) \in Q \times Q\ |\ \exists\
					\gamma_n \in \Gamma : ((q, R) \in (p, \gamma_n))
					\vee ((q, \gamma_n) \in (p, \gamma_n))\} $

				\item
					Nechť~$ R_\delta^+ $ je tranzitivní uzávěr
					relace~$ R_\delta $ vypočtený Warshallovým
					algoritmem.

				\item
					Nechť výstup algoritmu já dán predikátem~$ \varphi $
					definovaným následovně: \\
					$ \varphi : \neg\ (\exists\ p, q \in Q : p = q_0
					\wedge q = q_f \wedge (p, q) \in R_\delta^+) $
			\end{enumerate}
	\end{addmargin}
	\QEDB


	\section*{3.~příklad}

	Uvažujte jazyk $ L_{42} = \{\langle M \rangle\ |\ \text{TS}\ M $
	zastaví na některém vstupu tak, že páska bude obsahovat
	právě~42 ne-blankových symbolů$ \} $. Dokažte pomocí redukce,
	že~$ L_{42} $ je nerozhodnutelný. Uveďte ideu důkazu částečné
	rozhodnutelnosti~$ L_{42} $.

	\subsection*{Řešení:}

	\textbf{Důkaz, že jazyk~$ \boldsymbol{L_{42}} $ je nerozhodnutelný.} \\
	Provedeme důkaz redukcí z~problému zastavení Turingova stroje ($ HP $).
	\begin{itemize}
		\item
			Jazyk, který charakterizuje $ HP $ bude vypadat následovně: \\
			$ HP = \{\langle M \rangle \# \langle w \rangle\ |\ M $ je
			Turingův stroj takový, že na řetězci~$ w $ zastaví$ \} $,
			kde~$ \langle M \rangle $ je kód Turingova stroje~$ M $
			a~$ \langle w \rangle $ je kód řetězce~$ w $.

		\item
			Zadaný jazyk~$ L_{42} = \{ \langle M \rangle\ |\ M $ je
			Turingův stroj takový, že zastaví na některém vstupu tak,
			že páska bude obsahovat právě~42 ne-blankových symbolů$ \} $,
			kde~$ \langle M \rangle $ je kód Turingova stroje~$ M $, bude
			charakterizovat problém, který tento jazyk reprezentuje.

		\item
			Navrhneme redukci~$ \sigma : \{0, 1, \#\}^* \rightarrow
			\{0, 1\}^* $ z~jazyka $ HP $ na jazyk~$ L_{42} $.

		\item
			Redukce~$ \sigma $ přiřadí řetězci $ x \in \{0, 1, \#\}^* $
			řetězec $ \langle M_x \rangle $, což je kód Turingova
			stroje~$ M_x $, který pracuje následovně:

			\begin{enumerate}
				\item
					$ M_x $ smaže svůj vstup~$ w $.

				\item
					$ M_x $ zapíše na vstupní pásku řetězec~$ x $,
					který má uložen v~konečném stavovém řízení.

				\item
					$ M_x $ ověří, zda~$ x $ má strukturu $ x_1\#x_2 $,
					kde~$ x_1 $ je kód Turingova stroje a~$ x_2 $ je
					kód jeho vstupu. Pokud ne, odmítne.

				\item
					$ M_x $ odsimuluje na řetězci s~kódem~$ x_2 $ běh
					Turingova stroje s~kódem~$ x_1 $. Pokud simulace
					skončí, smaže svou pásku, zapíše na ni~42
					libovolných ne-blankových symbolů a~následně přijme.
					Jinak cyklí.
			\end{enumerate}

		\item
			Redukci~$ \sigma $ je možné implementovat úplným Turingovým
			strojem~$ M_\sigma $, který pro vstup~$ x $ vyprodukuje
			kód Turingova stroje~$ M_x $. Tento sestává z~následujících
			komponent:

			\begin{enumerate}
				\item
					Komponenta, která maže vstupní pásku\,---\,lze
					předpřipravit a~pak~$ M_\sigma $ jen vypíše
					patřičný kód.

				\item
					$ M_\sigma $ vypíše kód Turingova stroje, který
					jen zapíše na vstup řetězec $ a_1, a_2, \ldots,
					a_n $\,---\,opakovaný sekvenční zápis a~posuv
					doprava.

				\item
					$ M_\sigma $ vypíše kód Turingova stroje, který
					na vstupu ověří, zda se jedná o~platnou instanci
					$ HP $ a~pokud ne, odmítne. (Test na členství
					v~regulárním jazyce.)

				\item
					$ M_\sigma $ vypíše kód Turingova stroje, který
					spustí univerzální Turingův stroj na Turingův
					stroj s~kódem~$ x_1 $ a~vstupem s~kódem~$ x_2 $.
			\end{enumerate}

		\item
			$ M_\sigma $ zajistí sekvenční předávání řízení mezi
			jednotlivými komponentami.

		\item
			Nyní zkoumejme jazyk Turingova stroje~$ M_x $:

			\begin{enumerate}[a)]
				\item
					$ L(M_x) = \varnothing \Leftrightarrow ( x $
					nemá strukturu $ x_1\#x_2 $ pro kód Turingova
					stroje~$ x_1 $ a~kód vstupu~$ x_2) \vee ( x $
					má strukturu $ x_1\#x_2 $, kde~$ x_1 $ je kód
					Turingova stroje a~$ x_2 $ kód vstupu, ale
					Turingův stroj s~kódem~$ x_1 $ na vstupu
					s~kódem~$ x_2 $ nezastaví$ ) $.

				\item
					$ L(M_x) = \Sigma^* \Leftrightarrow ( x $
					má strukturu $ x_1\#x_2 $, kde~$ x_1 $ je kód
					Turingova stroje a~$ x_2 $ je kód vstupu
					a~Turingův stroj s~kódem~$ x_1 $ zastaví na
					vstupu s~kódem~$ x_2) \wedge (M_x $ má po
					přijetí na pásce právě~42 ne-blankových symbolů.
			\end{enumerate}

		\item
			Konečně ukážeme, že redukce~$ \sigma $ zachová členství
			v~jazyce: \\
			$ \forall\ x \in \{0, 1, \#\}^* : (\sigma(x) = \langle
			M_x \rangle \in L_{42}) \Leftrightarrow (L(M_x) =
			\Sigma^*) \Leftrightarrow ((x $ má strukturu $ x_1\#x_2 $,
			kde~$ x_1 $ je kód Turingova stroje a~$ x_2 $ kód vstupu$
			) \wedge ( $Turingův stroj s~kódem~$ x_1 $ zastaví na
			vstupu s~kódem~$ x_2) \wedge (M_x $ má po přijetí na
			pásce právě~42 ne-blankových symbolů$ )) \Leftrightarrow
			x \in HP $.
	\end{itemize}
	\QEDB

	\textbf{%
		Idea důkazu, že jazyk~$ \boldsymbol{L_{42}} $ je  částečně
		rozhodnutelný.
	} \\
	Lze sestavit Turingův stroj~$ M^\prime $ rozhodující
	jazyk~$ L_{42} $ následujícím způsobem:
	\begin{itemize}
		\item
			$ M^\prime $ zkontroluje, zda na vstupu má platný kód
			Turingova stroje~$ M $. Pokud ne, odmítne.

		\item
			$ M^\prime $ na pomocné pásce postupně simuluje běh
			Turingova stroje~$ M $ na jednotlivých vstupních
			řetězcích~$ w $. Jednotlivé páskové konfigurace jsou
			na pomocné pásce vhodně uspořádány.

		\item
			$ M^\prime $ vždy projde všechny rozpracované simulace
			a~na každé dále simuluje jeden krok výpočtu.

		\item
			Pokud byl v~některém kroce řetězec přijat, $ M^\prime $
			taky přijme. V~opačném případě se přidá pásková konfigurace
			pro další řetězec a~kroky simulace se opakují.
	\end{itemize}
	\QEDB


	\section*{4.~příklad}

	Uvažujte programovací jazyk \textbf{\texttt{Karel@TIN}}
	se~zadanou gramatikou a~sémantikou.

	Dokažte, že programovací jazyk \textbf{\texttt{Karel@TIN}} je
	Turingovksy úplný, tj., dokažte, že
	\begin{enumerate}[(a)]
		\item
			pro každý TS~$ M $ nad abecedou $ \{0, 1\} $ a~řetězec
			$ w \in \{0, 1\}^* $ lze sestrojit program~$ P_M $
			v~jazyce \textbf{\texttt{Karel@TIN}} a~zvolit
			počáteční konfiguraci prostředí~$ C_M $ tak, že~$ P_M $
			skončí s~návratovou hodnotou~1 právě tehdy, když
			$ w \in L(M) $;

		\item
			pro každý program~$ P $ v~jazyce \textbf{\texttt{Karel@TIN}}
			a~počáteční konfiguraci~$ C $ lze spustit TS~$ M_P $
			a~řetězec $ w \in \{0, 1\}^* $ tak, že $ w \in L(M_P) $
			právě tehdy, když robot Karel po interpretaci programu~$ P $
			z~počáteční konfigurace~$ C $ skončí s~návratovou
			hodnotu~1.
	\end{enumerate}

	\subsection*{Řešení:}

	\begin{enumerate}[(a)]
		\item
		    Počáteční konfigurace~$ C_M = ((0, 0), \uparrow, g) $. \\
		    Kódování symbolů: $ 0 \backsimeq 0 $, $ 1 \backsimeq 1 $,
		    $ \Delta \backsimeq 2 $.

            \texttt{write\_blank}:
            \begin{lstlisting}
if empty: goto 3;
lift-screw;
if not empty: goto 1;
drop-screw;
drop-screw;
            \end{lstlisting}
            
            \texttt{write\_0}:
            \begin{lstlisting}
if empty: goto 3;
lift-screw;
if not empty: goto 1;
turn left;
turn left;
turn left;
turn left;
            \end{lstlisting}
            
            \newpage
            \texttt{write\_1}:
            \begin{lstlisting}
if empty: goto 3;
lift-screw;
if not empty: goto 1;
drop-screw;
            \end{lstlisting}
            
            \texttt{move\_forward}:
            \begin{lstlisting}
step;
            \end{lstlisting}
            
            \texttt{move\_backward}:
            \begin{lstlisting}
turn left;
turn left;
step;
turn left;
turn left;
            \end{lstlisting}
	\end{enumerate}
\end{document}
